\section{Design og implementering}

Syntetiseringen bygger på et valg af grundfrekvens \(f_0\) for den tone der skal afspilles.
Ud fra den almindelig tonerække for en stemt guitar\cite{GuitarTunings}, kan der nemt vælges et sæt frekvenser til test, disse er gengivet i \tabref{guitar tunings}.
Sammenhængen mellem grundfrekvens og samplingfrekvens ses i \eqref{eq:fund and sample relation} og giver tonens periodetid \(N\) i samples og kan desuden beskrives som en forsinkelse i det filter der realiserer algoritmen.

\begin{equation}\label{eq:fund and sample relation}
  N = \frac{f_s}{f_0}
\end{equation}

På \figref{Karplus-strong-schematic} ses blokdiagram der beskriver filteret hvor \(L = N\). I den originale algoritme bruges et 2-tap midlingsfilter da algoritmen blev udviklet til at kræve minimal beregningskraft\cite{Karplus1983}, dog kan andre filtre også bruges.

\fig{Karplus-strong-schematic}{0.4}{
  Signalflow der viser konceptet af Karplus Strong syntese.\cite{KarplusSignalFlowWiki}}

Jaffe \& Smith beskriver algoritmen med differensligningen
\begin{equation}\label{eq:JaffeString}
  y[n] = x[n] + \frac{y[n - N] + y\left[n -(N + 1)\right]}{2}
\end{equation}
hvor \(x\) er signalamplituden ved sample \(n\) og \(y\) er udgangsamplitude ved sample \(n\).
\(N\) er længden af toneperioden givet i samples\cite{Jaffe1983}.

\begin{table}[]
  \centering
  \caption{Guitar streng og sammenhørende frekvens ved almindelig stemning\cite{GuitarTunings}.}
  \label{tab:guitar tunings}
  \begin{tabular}{cc}
    \toprule
    Streng & Frekvens  \\ \midrule
    E      & 329.63 Hz \\
    B      & 246.94 Hz \\
    G      & 196.00 Hz \\
    D      & 146.83 Hz \\
    A      & 110.00 Hz \\
    E      & 82.41 Hz  \\ \bottomrule
  \end{tabular}
\end{table}

\subsection{Implementation af Karplus-Strong algoritmen for en streng}
Udfra
\begin{equation}
  f_0 = \frac{F_s}{p + \frac{1}{2}},
\end{equation} er det givet at grundtonen der skal afspilles har en sammenhæng med $p$ værdien samt sampleraten\cite{Karplus1983}. Udfra dette kan \coderef[ToneSet]{KSPluck.m} oprettes.
\includecode[ToneSet]{KSPluck.m}{}{linerange={ToneStart-ToneEnd}}
Der tages forbehold for at $p \in \mathbb{Z^+}$ ved at afrunde resultatet.

\subsubsection{Begyndelsesbetingelser}
For at kunne genere en række forskellige overtoner, er det nødvendigt at have en række tilfældige samples indlæst i en \emph{wavetable}. Dette er et array med $p$ pladser, hvori der ligger en række samples. Disse samples bliver gentaget gang på gang i løbet af algoritmen, hvorved der opstår en række gentagelser.
\begin{equation}
  x_{n} =
  \begin{cases}
    A u_n & n = 0,1,2 \dots , p - 1 \\
    0,    & n \geq p,
  \end{cases}
  \quad u_n \in [-1:1]
\end{equation}


I \coderef[Initial]{KSPluck.m} ses opsætningen af \emph{intial conditions} der bruges i strengalgoritmen set i \eqref{eq:JaffeString}. Begyndelsesværdierne kan beskrives \cite{Jaffe1983}

\includecode[Initial]{KSPluck.m}{Opsætning af \emph{initial conditions} for strengalgoritmen.}{linerange={InitialStart-InitialEnd}}

\texttt{randi([0 1], [1, PValue]) }giver en tilfældig sekvens af \texttt{0 0 1 0 1 1 0 1}. \emph{Element-wise} ganges denne med 2 så det bliver \texttt{0 0 2 0 2 2 0 2}. Derefter trækkes \texttt{1} fra så den resulterende sekvens bliver \texttt{-1 -1 1 -1 1 1 -1 1} hvorved $u_n$ er oprettet. Amplituden $A$ holdes på værdien 1.

Resultatet er en række tilfældige værdier i et array med størrelsen p. Slutteligt ganges en række nuller på for at kunne styre længden af output lyden. \texttt{Duration} er her en værdi opgivet i sekunder, hvorved en samplelængde kan indstilles uafhængigt af sampleraten.
\subsubsection{\emph{2-tap } midling af samples}
\includecode[Pluck]{KSPluck.m}{Selve midlingsfilteret i aktion. Bemærk \texttt{PrevValue=0}, dette gøres for at undgå $y(n-p-1)$ ved første gennemløb, da dette vil give negative et negativt indeks. Derefter ersattes \emph{PrevValue} med den netop udregnede værdi.}{linerange={PluckStart-PluckEnd}}

\fig{Offset.png}{0.6}{\emph{Skitse} der viser forskydningen af metoden. Der indlæses en række værdier i området $0 \le n \le p$, derefter sættes startpunktet for algortimen til $p$, hvorved området $0 \le n \le p$ fungere som et \emph{initial conditions.}. Her ses for \(u_n \in [0:1]\)}

TIl sidst laves en række justeringer på outputtet, hvilket ses i \coderef[Output]{KSPluck.m}.
\includecode[Output]{KSPluck.m}{Output-stadie hvori: Startværdien sættes til $p$, DC offset fjernes og normalisering af lyden så maksimal værdiern bliver 1. Tilsidst ses returværdien \texttt{KSPluckSound}}{linerange={OutputStart-FuncEnd}}

Ovenstående kan herefter kaldes ved hjælp af \texttt{KSPluck(Tone,Duration,SampleRate)} og strengalgoritmen er dermed implementeret.

Den fulde implementation ses i bilag \ref{seq:appendix:kspluck}.


\subsection{Implementation af Karplus-Strong algoritmen for trommer}

Som beskrevet i den originale artiklen\cite{Karplus1983} kan strengalgoritmen udvides til at give trommelyde, ved at tilføje et tilfældighedselement i filteret.
Baseret på implementationen af strengalgoritmen \texttt{KSPluck} udvides filtret i en ny funktion, således at trommelyde kan syntetiseres.
Tilfældigheden \textit{b} kaldes for \textit{blend faktor} og styrer fortegnet for den resulterende udgangsamplitude når samples føres igennem filteret.
Trommealgoritmen beskrives med \eqref{eq:drum algo}.

\begin{equation}\label{eq:drum algo}
  y[n] =
  \begin{dcases*}
    +\frac{y[n-N] + y[n - (N+1)]}{2} & sandsynlighed \(b\)   \\
    -\frac{y[n-N] + y[n - (N+1)]}{2} & sandsynlighed \(1-b\)
  \end{dcases*}
\end{equation}

Da trommealgoritme i sig selv indeholder tilfældighed kan begyndelsesbetingelsen sættes til en konstant værdi, hvilket ses i \coderef[Initial]{KSDrum.m} hvor output fyldes med 1-taller som startværdi.
Desuden genereres et antal tilfældige værdier i intervallet \([0; 1]\) som bruges til at skifte fortegn baseret på \textit{blend faktoren}.

\includecode[Initial]{KSDrum.m}{Begyndelsesbetingelser for trommealgoritmen}{linerange=InitialStart-InitialEnd}

Selve filtreringen er næsten magen til strengalgoritmen, dog med en \texttt{if-else} til at skifte fortegn baseret på de genererede tilfældige værdier og den specificerede \textit{blend faktor}.
Filtreringen ses i \coderef[Filtering]{KSDrum.m}.

\includecode[Filtering]{KSDrum.m}{Filtrering der skaber en trommelyd}{linerange=DrumStart-DrumEnd}

Den fulde implementering ses i \ref{seq:appendix:ksdrum} og indeholdes i funktionen \texttt{KSDrum(PValue, b, Duration, Fs)} hvor det påpeges at funktionsforskriften er en smule anderledes end for \texttt{KSPluck}.
Dette fordi det ikke giver mening at angive en \textit{Tone} for tromme algoritmen, i stedet angives \textit{P}-værdien direkte, da denne styrer lydens aftagelseshastighed.