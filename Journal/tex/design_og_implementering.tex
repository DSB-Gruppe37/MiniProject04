\section{Design og implementering}

Syntetiseringen bygger på et valg af grundfrekvens \(f_0\) for den tone der skal afspilles. Ud fra den almindelig tonerække for en stemt guitar\cite{GuitarTunings}, kan der nemt vælges et sæt frekvenser mellem \SI{82.41}{\hertz}--\SI{329.63}{\hertz} som kan afprøves for at vurdere algoritmens evne til at syntetisere korrekte toner.
Sammenhængen mellem grundfrekvens og samplingfrekvens ses i \ref{eq:fund and sample relation} og giver tonens periodetid \(N\) i samples og kan desuden beskrives som en forsinkelse i det filter der realiserer algoritmen.

\begin{equation}\label{eq:fund and sample relation}
  N = \frac{f_s}{f_0}
\end{equation}

På \figref{Karplus-strong-schematic} ses blokdiagram der beskriver filteret hvor \(L = N\). I den originale algoritme bruges et 2-tap midlingsfilter da algoritmen blev udviklet til at kræve minimal beregningskraft\cite{Karplus1983}, dog kan andre filtre også bruges.

\fig{Karplus-strong-schematic}{0.4}{
  Signalflow der viser konceptet af Karplus Strong syntese.\cite{KarplusSignalFlowWiki}}

Jaffe \& Smith beskriver algoritmen
\begin{equation}
  y[n] = x[n] + \frac{y[n - N] + y\left[n -(N + 1)\right]}{2}
\end{equation}
hvor \(x\) er signalamplituden ved sample \(n\) og \(y\) er udgangsamplitude ved sample \(n\).
\(N\) er længden af toneperioden givet i samples \cite{Jaffe1983}
