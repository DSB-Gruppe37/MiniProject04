\section{Diskussion}

\subsection{Strengalgoritmen}

Ved sammenligning af de gængse frekvenser for en stemt guitar i \tabref{guitar tunings} og de syntetiserede toners frekvenser som set i \figref{Pluck_FFTComparison} stemmer grundtonen overens med den tone der bliver angivet til syntesen.
Der ses desuden forskellige mængder af overtoner som også er forventeligt ved anslag på en rigtig guitarstreng.

Grundtonen er ikke nødvendigvis den dominerende, men dette stemmer desuden overens med en rigtig guitarstreng som set på sammenligningerne med en nylonstrengs guitar (f.eks. i \figref{Pluck_Compare_A2_FFT}), hvor grundtonen ligeledes heller ikke dominerer.
Dog gælder det for alle de syntetiserede toner at energien hurtigt forsvinder i de høje frekvenser.
Ved gennemlytning lyder de syntetiserede toner desuden meget som almindelige guitarstrenge, hvilket underbygger at algoritmen giver en realistisk lyd.

Ved sammenligning af syntetiserede toner med toner fra en nylonstrengsguitar, som set i bilag \ref{seq:appendix:results:pluck guitar compare} stemmer alle de fremtrædende frekvenser overens for begge typer toner.
Der er dog to ting der springer i øjnene ved sammenligning:

\begin{itemize}
  \item Der er væsentlig mere energi i de syntetiserede toner
  \item Der er væstenlig flere fremtrædende overtoner i de syntetiserede toner
\end{itemize}

Forskellen i energi kan skyldes måden hvorpå lyder fra guitaren er blevet optaget. Desuden burde amplituden af de to signaler kunne normaliseres, så de matcher bedre.
Hvad angår overtoner, så fremgår der i de syntetiserede væsentlig flere overtoner i de høje frekvenser.
Ved sammenligning af lyden er de syntetiserede toner også væsentlig mere metalliske.

\subsection{Trommealgoritmen}

\subsubsection{Blendfaktoren og sampleantal}


\Figref{DrumPValueComparison} viser hvordan $p$-værdien har en tydelig indvirkning i længden af lyden, hvor det ses at en høj $p$-værdi giver en lang tone.

I tidsdomænet,\figref{DrumPValueComparison} ses det tydeligt at $b=0$ producere, med undtagelse af $p=32$, et periodisk signal med form som et firkantsignal hvor $P_W = p$, dette signal illustrere samtidigt størrelsen den brugte \emph{wavetable}. Ved $b=1$ ses det at der også er et gentagende signal, der dog har en dæmpet amplitude.

$b=1/4 \; \& \; 3/4$ ligner i tidsdomænet hinanden, men har i frekvensdomænet,\figref{DrumFFT_p_128}, to forskellige udtryk. Det ses hvordan energiniveauet stiger i frekvensområdet \SI{>1000}{Hz} når $b \to 1$ og praktisktalt er væk ved $b=0$. Interessant er det at se hvordan en blendfaktor på 1 giver et næsten identisk resultat som ved brug af strengalgoritmen, hvilket ses i frekvensdomænet hvor der er tydelige overtoner tilstede. Dette skyldes sammenhængen set i \eqref{eq:drum algo}, hvor en blendfaktor vil få funktionen til at opføre sig som strengalgoritmen. Den eneste forskel ligger nu i begyndelsebetingelserne.

Ved gennemlytning af lydene bemærkes det at, hvis $p$-værdien øges tilstrækkeligt, bliver lydklangen meget metallisk at høre på, hvilket også ses i FFT graferne hvor det højfrekvente område bliver kraftigere og kraftigere i takt med en forøgelse af $p$.

Ved at justere $p$ og $b$ værdien er det dermed muligt at opnå en række forskellige klangudtryk.